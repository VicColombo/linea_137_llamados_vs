\documentclass[10 pt]{article}
\usepackage[a4paper, total={7in, 9.5in}]{geometry}
\usepackage[spanish]{babel}
\usepackage{csquotes}
\usepackage{graphicx} 
\usepackage{caption}
\usepackage{hyperref}
\usepackage{graphicx}
\usepackage{float}
\usepackage{url}
\usepackage{color}
\usepackage{siunitx}
\usepackage{hhline}
\usepackage{multirow}
\usepackage{natbib}
\bibliographystyle{plainnat}
\usepackage{todonotes}

\begin{document}

\begin{titlepage}

    \begin{center}
    \vspace*{-0.5in}
    \begin{figure}[htb]
    \begin{center}
    \includegraphics[scale=.3]{images/uba2.jpg}
    \end{center}
    \end{figure}
    
    \begin{large}
    Maestría en Explotación de Datos y Descubrimiento del Conocimiento\\
    \vspace*{0.15in}
    Universidad de Buenos Aires \\
    
    \vspace*{0.6in}
    \end{large}
    
    \begin{large}
    TÍTULO DEL TRABAJO\\
    
    
    \end{large}
    \vspace*{0.2in}
    \vspace*{0.3in}
    
    
    \vspace*{0.3in}
    \rule{80mm}{0.1mm}\\
    \vspace*{0.1in}
    \begin{large}
    Victoria Colombo
    
    \vspace*{0.3in}
    
    \vspace*{0.1in}fecha
    \end{large}
    \end{center}
    
    \end{titlepage}

\newpage

\begin{abstract}

\end{abstract}
\newpage
\section*{Introducción}\label{intro}


La violencia sexual es todo acto o intento de acto sexual, o comentarios o insinuaciones sexuales no deaseados, intentados o consumados contra la sexualidad de otra persona de manera coercitiva, por cualquier persona sin importar el vínculo con la víctima, y en cualquier ámbito, incluyendo el hogar y el trabajo \citep*{ferris2002world}. Estudios sobre violencia sexual a nivel mundial demuestran que las víctimas de violencia sexual son mayormente mujeres y los perpetradores mayormente hombres, y además, que los perpetradores son en la amplia mayoría de los casos conocidos por las víctimas, no extraños \citep*{garcia2005multi,contreras2016violencia}. A nivel local, en esta misma línea, el más reciente informe nacional \textit{Relevamiento de fuentes secundarias de datos sobre violencia sexual} de la UFEM \citep*{ufem_relevamiento} especifica que la violencia sexual “afecta particularmente a las mujeres cis y personas LGBTI+“ (p.7). Estas estadísticas se ven reflejadas   




la situaci;on es m;as compleja, no? menos denuncias de hombres, menos denuncas de minor;ias no cis género excede el trabajo por los tiposde datos que tengo
Por ejemplo durante la pandemia los números de violencia de género específicamente contra las mujeres no bajaron subieron estaban en sus casas.


La  recopilación, sistematización, y análisis de datos sobre violencia sexual por parte de los Estados es crucial para planificar y llevar adelante políticas efectivas de prevención, asistencia, y erradicación de la violencia sexual. En Argentina no hay un sistema estatal único y centralizado de este tipo de información. Sin embargo, existen entidades judiciales y programas estatales que, además de ofrecer auxilio, asistencia y/o acceso a la justicia, recaban datos sobre violencia sexual, y mantienen un registro público de ellos en el Portal de Datos Abiertos del Ministerio de Justicia. 

En este trabajo en paticular analizaré registros provenientes del programa Las Víctimas contra las Violencias, dependiente del Ministerio de Justicia y Derechos Humanos de la Nación, fue creado en el año 2006 con el objetivo de brindar atención e intervención institucional a víctimas de abusos y violencia familiar o sexual. Para solicitar asistencia, acompañamiento, y asesoramiento para el acceso a la justicia, las víctimas cuentan, desde 2016, con la línea nacional de emergencia 137 que funciona las 24 horas del día, todo el año, y cuenta en al menos cinco ciudades del país con equipos especializados para llevar a cabo el acompañamiento y las intervenciones necesarias. Los registros de esas llamadas e intervenciones se encuentran digitalizados al menos desde 2017 y están disponibles en el \href{http://datos.jus.gob.ar/}{Portal de Datos Abiertos de la Justicia Argentina}. Allí se encuentran dos tipos de \textit{datasets}: llamados e intervenciones domiciliarias por situaciones de violencia familiar, y llamados e intervenciones domiciliarias por situaciones de violencia sexual. \todo{rever cómo sigue el programa ahora}



\subsection*{objetivo}
- objetivos: imputar los NS/NC de convive no convive con probabilidad de SI/NO 
El dataset tiene muchos datos faltantes, como se verá en la sección \nameref{datos} a continuación.


\section*{Datos}\label{datos}

Para este trabajo he tomado en principio los llamados de denuncias por violencia sexual desde enero de 2017 hasta julio de 2021. El \textit{dataset} se compone en total de 19143 observaciones y 54 variables, en su mayoría categóricas, que aportan información sobre la víctima, la persona denunciante, el contexto del hecho y el tipo de violencia sufrida. En la tabla \ref{tablavar} se detallan las variables y su tipo.

\begin{table}[ht!]
    \centering
    \caption{Resumen de las variables.}
    \label{tablavar}
    \begin{tabular}{|l|l|l|l|} 
    \hline
    Descriptor & Tipo variable & Variables & Observaciones         
    \\ 
    \hline
    \multirow{2}{*}{Víctima} & Cuantitativa & victima\_edad &  
    \\ 
    \cline{2-4}
     & Cualitativa & \begin{tabular}[c]{@{}l@{}}victima\_genero, victima\_nacionalidad, \\ victima\_discapacidad,\\ victima\_vinculo\_agresor,\\victima\_convive\_agresor,\\ victima\_a\_resguardo\end{tabular} & \begin{tabular}[c]{@{}l@{}}victima\_genero toma los valores: \\masculino, femenino, trans, \\ NS/NC.\end{tabular}           
     \\ 
    \hline
    \multirow{2}{*}{Llamante} & Cuantitativa         & {llamante\_edad}&               \\ 
    \cline{2-4}
     & Cualitativa& llamante\_genero, llamante\_vinculo & \begin{tabular}[c]{@{}l@{}}llamante\_genero toma los valores: \\masculino, femenino, trans, \\ NS/NC.\\llamante\_vinculo\_ refiere a vínculo \\con la víctima.\end{tabular}                \\ 
    \hline
    \multirow{2}{*}{Llamado}& Cuantitativa & llamado\_fecha\_hora&                                             \\ 
    \cline{2-4} & Cualitativa & \begin{tabular}[c]{@{}l@{}} caso\_id, llamado\_provincia, \\ llamado\_provincia\_id, \\ caso\_judicializado, hecho\_lugar\end{tabular} & \begin{tabular}[c]{@{}l@{}} llamado\_provincia\_id refiere al id \\numérico para las provincias \\según codificación INDEC.\end{tabular}                                \\ 
    \hline
    Violencia sexual                 & Cualitativa  &  \begin{tabular}[c]{@{}l@{}} vs\_violacion\_via\_vaginal, \\ vs\_violacion\_via\_anal, \\ vs\_violacion\_via\_oral, \\ vs\_tentativa\_violacion, \\ vs\_tocamiento\_sexual, \\ vs\_intento\_tocamiento, \\ vs\_intento\_violacion\_tercera\_persona, \\ vs\_grooming, vs\_exhibicionismo, \\ vs\_amenazas\_verbales\_contenido\_sexual, \\ vs\_explotacion\_sexual, \\ vs\_explotacion\_sexual\_comercial, \\ vs\_explotacion\_sexual\_viajes\_turismo,\\ vs\_sospecha\_trata\_personas- \\ \_fines\_sexuales, \\ vs\_existencia\_facilitador-\\
    \_corrupcion\_nnya, \\ vs\_obligacion\_sacarse\_fotos\_pornograficas, \\ vs\_eyaculacion\_partes\_cuerpo, \\ vs\_acoso\_sexual, \\ vs\_iniciacion\_sexual\_forzada\_inducida, \\ vs\_otra\_forma\_violencia\_sexual, \\ vs\_no\_sabe\_no\_contesta \end{tabular} &
    
    \begin{tabular}[b]{@{}l@{}} vs\_existencia\_facilitador-\\
    \_corrupcion\_nnya \\ refiere a la existencia de un \\ facilitador de la corrupción de \\ niños, niñas y adolescentes.\\\\ vs\_no\_sabe\_no\_contesta refiere \\ violencia sexual que se desconoce \\ o que no hace referencia a los otros \\ campos mencionados.\end{tabular}                                                  \\ 
    \hline
    Otras violencias & Cualitativa & \begin{tabular}[c]{@{}l@{}} ofv\_sentimiento\_amenaza, \\ ofv\_amenazas\_explicitas, \\ ofv\_violencia\_fisica, ofv\_intento\_ahorcar, \\ ofv\_intento\_quemar,  ofv\_intento\_ahogar, \\ ofv\_amenaza\_muerte, \\ ofv\_uso\_sustancias\_psicoactivas, \\ ofv\_intento\_privacion\_libertad, \\ ofv\_privacion\_libertad, \\ ofv\_uso\_arma\_blanca, \\ ofv\_uso\_arma\_fuego, \\ ofv\_enganio\_seduccion, ofv\_intento\_matar, \\ ofv\_uso\_animal\_victimizar, \\ ofv\_grooming, ofv\_otra\_forma\_violencia, \\ ofv\_no\_sabe\_no\_contesta\end{tabular} &  \\ 
    \hline
    \end{tabular}
    \end{table}
    

\section*{Metodología}\label{met}

\section*{Resultados}\label{red}

\section*{Discusión y conclusiones}\label{conc}
cruzamiento de datos ovd líneas de asistencia, observatorio de género.
acceso y análisis de datos extensivo a provincias, no solo benos aitres

\newpage


\bibliography{bibtex_reporte.bib}

\newpage
\section*{Anexo}\label{anex}

\end{document}