\documentclass[10 pt]{article}
\usepackage[a4paper, total={7in, 9.5in}]{geometry}
\usepackage[spanish]{babel}
\usepackage{csquotes}
\usepackage{graphicx} 
\usepackage{caption}
\usepackage{hyperref}
\usepackage{graphicx}
\usepackage{float}
\usepackage{url}
\usepackage{color}
\usepackage{siunitx}
\usepackage{hhline}
\usepackage{multirow}
\usepackage[round]{natbib}
\bibliographystyle{plainnat}
\usepackage{todonotes}
\linespread{1.5}

\begin{document}

\begin{titlepage}

    \begin{center}
    \vspace*{-0.5in}
    \begin{figure}[htb]
    \begin{center}
    \includegraphics[scale=.3]{images/uba2.jpg}
    \end{center}
    \end{figure}
    
    \begin{large}
    Maestría en Explotación de Datos y Descubrimiento del Conocimiento\\
    \vspace*{0.15in}
    Universidad de Buenos Aires \\
    
    \vspace*{0.6in}
    \end{large}
    
    \begin{large}
    TÍTULO DEL TRABAJO\\
    
    
    \end{large}
    \vspace*{0.2in}
    \vspace*{0.3in}
    
    
    \vspace*{0.3in}
    \rule{80mm}{0.1mm}\\
    \vspace*{0.1in}
    \begin{large}
    Victoria Colombo
    
    \vspace*{0.3in}
    
    \vspace*{0.1in}fecha
    \end{large}
    \end{center}
    
    \end{titlepage}

\newpage

\begin{abstract}

\end{abstract}
\newpage
\section*{Introducción}\label{intro}


La violencia sexual comprende una multiplicidad de conductas o intentos de conductas, que van desde actos hasta comentarios sexuales, dirigidos contra la sexualidad de otra persona de manera coercitiva. El trabajo con datos sobre violencia sexual presenta complicaciones porque los datos suelen ser escasos o presentar gran cantidad de faltantes \citetext{\citealp[p.~150]{ferris2002world}}. Uno de los motivos es que las víctimas o su entorno a menudo se rehúsan a denunciar o participar en encuestas sobre este tipo de agresiones, o proveen información incompleta. Esto puede deberse a la vergüenza y el estigma social frecuentemente asociado no solo con la violencia sexual sino con la sexualidad en general, pero también a la falta de acceso a la justicia, al temor a las represalias por parte de los agresores, o el temor a que la denuncia no sea creída \citep*{murphy2022unfounded}. Otros posibles motivos para la escasez y/o mala calidad de los datos pueden ser la falta de vías adecuadas para recabar esta información, o la negligencia o desconocimiento de procedimientos adecuados por parte de oficiales de policía encargados de recibir denuncias. A pesar de las dificultades en la recolección de datos, diversos estudios a nivel mundial logran identificar patrones frecuentes en la violencia sexual. Para este trabajo, resultan relevantes dos de ellos: la mayoría de las víctimas son mujeres, mientras que los perpetradores suelen ser hombres \citetext{\citealp[p.~149]{ferris2002world}; \citealp[p.~15]{contreras2016violencia}}; y en la mayoría de los casos, los agresores son personas conocidas por las víctimas, como parejas, exparejas u otros conocidos \citetext{\citealp[p.~9]{garcia2005multi},\citealp[p.~22]{unicef2018analisis}, \citealp[p.~151]{ferris2002world}}.
 
La clasificación de las identidades de género de víctimas y perpetradores es compleja. Por un lado, muchos estudios clasifican a las personas únicamente como hombres o mujeres, omitiendo las identidades de género disidentes.\footnote{Entre los estudios e informes consultados para este trabajo, solamente el \textit{Relevamiento de fuentes secundarias de datos sobre violencia sexual} de la \citet{ufem_relevamiento} menciona identidades de género cuando especifica que la violencia sexual “afecta particularmente a las mujeres cis y personas LGBTI+” (p.7).}. Por otro lado, aunque se reportan pocos casos de violencia sexual contra hombres cisgénero, es probable que estén subrepresentados debido a los prejuicios y estigmas sociales sobre la masculinidad que dificultan las denuncias y el acceso a la justicia para estas víctimas \citep*[p.~149]{ferris2002world}. Analizar esas complejidades excede a este trabajo de especialización. En mi análisis las categorías de género de víctimas, victimarios y llamantes se limitan a las registradas en el \textit{dataset}: hombre, mujer, y transgénero, sin especificar si es un hombre o una mujer transgénero. Reconozco esto como una limitación no solo de mi trabajo sino también de los datos disponibles.  

La  recopilación, sistematización, y análisis de datos sobre violencia sexual por parte de los Estados es crucial para planificar y llevar adelante políticas efectivas de prevención, asistencia, y erradicación de la violencia sexual. En Argentina, si bien no hay un sistema estatal único y centralizado de este tipo de información, existen entidades judiciales y programas estatales que, además de ofrecer auxilio, asistencia y/o acceso a la justicia, recaban datos sobre violencia sexual, y mantienen un registro público de ellos. Unos de esos programas es Las Víctimas contra las Violencias. 

Desde el año 2016, en el marco del programa Las Víctimas contra las Violencias, dependiente del Ministerio de Justicia de la Nación, la línea 137 funciona las 24 horas del día para solicitar asistencia en casos de violencia sexual o familiar\footnote{Además, desde 2020 cuenta también con el canal de \textit{Whatsapp} (54911) 3133-1000.}. El programa cuenta con equipos de intervención de abogadas, psicólogas, y trabajadoras sociales. Al recibir una llamada solicitando asistencia se coordina el envío de equipos móviles para proveer a la víctima, en base a las necesidades del caso, de contención emocional, acompañamiento a un hospital y/o a radicar una denuncia, y/o a un lugar seguro donde pueda alojarse \citep*{linea_137}.\todo{rever cómo sigue el programa ahora}

Los registros de las llamadas a la línea 137 se encuentran digitalizados desde 2017 y están disponibles en el \href{http://datos.jus.gob.ar/}{Portal de Datos Abiertos de la Justicia Argentina}. Allí se encuentran publicados cuatro tipos de \textit{datasets} por año: llamados e intervenciones domiciliarias por situaciones de violencia familiar, y llamados e intervenciones domiciliarias por situaciones de violencia sexual.
Los registros no están exentos de los problemas frecuentes antes mencionados en los datos sobre violencia sexual, ya que presentan información faltante de dos maneras: celdas vacías en el caso de las variables numéricas de edad, y respuestas NS/NC (no sabe-no contesta) en lugar de SI o NO en el resto de las variables categóricas. Teniendo en cuenta la clasificación de datos faltantes que se origina en \citet{rubin1976inference}, los datos faltantes en el \textit{dataset} de llamados son posiblemente del tipo \textit{missing at random} (MAR) y \textit{missing not at random/ non-ignorable missing data} (MNAR). Es decir, o bien los datos faltan por motivos que tienen que ver con otras variables (MAR), o bien el valor de los datos que faltan está relacionado con el motivo mismo por el que faltan (MNAR).

En este trabajo analizo llamados para reportar violencia sexual a la línea 137 entre 2017 y 2021, e intento predecir valores faltantes de la variable “víctima convive con el agresor”. 



\section*{Datos}\label{datos}


Para este trabajo he tomado en principio los llamados de denuncias por violencia sexual desde enero de 2017 hasta julio de 2021. El \textit{dataset} se compone en total de 19143 observaciones y 54 variables, en su mayoría categóricas, que aportan información sobre la víctima, la persona denunciante, el contexto del hecho y el tipo de violencia sufrida. En la tabla \ref{tablavar} se detallan las variables y su tipo.

\begin{table}[ht!]
    \centering
    \caption{Resumen de las variables.}
    \label{tablavar}
    \begin{tabular}{|l|l|l|l|} 
    \hline
    Descriptor & Tipo variable & Variables & Observaciones         
    \\ 
    \hline
    \multirow{2}{*}{Víctima} & Cuantitativa & victima\_edad &  
    \\ 
    \cline{2-4}
     & Cualitativa & \begin{tabular}[c]{@{}l@{}}victima\_genero, victima\_nacionalidad, \\ victima\_discapacidad,\\ victima\_vinculo\_agresor,\\victima\_convive\_agresor,\\ victima\_a\_resguardo\end{tabular} & \begin{tabular}[c]{@{}l@{}}victima\_genero toma los valores: \\masculino, femenino, trans, \\ NS/NC.\end{tabular}           
     \\ 
    \hline
    \multirow{2}{*}{Llamante} & Cuantitativa         & {llamante\_edad}&               \\ 
    \cline{2-4}
     & Cualitativa& llamante\_genero, llamante\_vinculo & \begin{tabular}[c]{@{}l@{}}llamante\_genero toma los valores: \\masculino, femenino, trans, \\ NS/NC.\\llamante\_vinculo\_ refiere a vínculo \\con la víctima.\end{tabular}                \\ 
    \hline
    \multirow{2}{*}{Llamado}& Cuantitativa & llamado\_fecha\_hora&                                             \\ 
    \cline{2-4} & Cualitativa & \begin{tabular}[c]{@{}l@{}} caso\_id, llamado\_provincia, \\ llamado\_provincia\_id, \\ caso\_judicializado, hecho\_lugar\end{tabular} & \begin{tabular}[c]{@{}l@{}} llamado\_provincia\_id refiere al id \\numérico para las provincias \\según codificación INDEC.\end{tabular}                                \\ 
    \hline
    Violencia sexual                 & Cualitativa  &  \begin{tabular}[c]{@{}l@{}} vs\_violacion\_via\_vaginal, \\ vs\_violacion\_via\_anal, \\ vs\_violacion\_via\_oral, \\ vs\_tentativa\_violacion, \\ vs\_tocamiento\_sexual, \\ vs\_intento\_tocamiento, \\ vs\_intento\_violacion\_tercera\_persona, \\ vs\_grooming, vs\_exhibicionismo, \\ vs\_amenazas\_verbales\_contenido\_sexual, \\ vs\_explotacion\_sexual, \\ vs\_explotacion\_sexual\_comercial, \\ vs\_explotacion\_sexual\_viajes\_turismo,\\ vs\_sospecha\_trata\_personas- \\ \_fines\_sexuales, \\ vs\_existencia\_facilitador-\\
    \_corrupcion\_nnya, \\ vs\_obligacion\_sacarse\_fotos\_pornograficas, \\ vs\_eyaculacion\_partes\_cuerpo, \\ vs\_acoso\_sexual, \\ vs\_iniciacion\_sexual\_forzada\_inducida, \\ vs\_otra\_forma\_violencia\_sexual, \\ vs\_no\_sabe\_no\_contesta \end{tabular} &
    
    \begin{tabular}[b]{@{}l@{}} vs\_existencia\_facilitador-\\
    \_corrupcion\_nnya \\ refiere a la existencia de un \\ facilitador de la corrupción de \\ niños, niñas y adolescentes.\\\\ vs\_no\_sabe\_no\_contesta refiere \\ violencia sexual que se desconoce \\ o que no hace referencia a los otros \\ campos mencionados.\end{tabular}                                                  \\ 
    \hline
    Otras violencias & Cualitativa & \begin{tabular}[c]{@{}l@{}} ofv\_sentimiento\_amenaza, \\ ofv\_amenazas\_explicitas, \\ ofv\_violencia\_fisica, ofv\_intento\_ahorcar, \\ ofv\_intento\_quemar,  ofv\_intento\_ahogar, \\ ofv\_amenaza\_muerte, \\ ofv\_uso\_sustancias\_psicoactivas, \\ ofv\_intento\_privacion\_libertad, \\ ofv\_privacion\_libertad, \\ ofv\_uso\_arma\_blanca, \\ ofv\_uso\_arma\_fuego, \\ ofv\_enganio\_seduccion, ofv\_intento\_matar, \\ ofv\_uso\_animal\_victimizar, \\ ofv\_grooming, ofv\_otra\_forma\_violencia, \\ ofv\_no\_sabe\_no\_contesta\end{tabular} &  \\ 
    \hline
    \end{tabular}
    \end{table}
    

\section*{Metodología}\label{met}

\section*{Resultados}\label{red}

\section*{Discusión y conclusiones}\label{conc}
cruzamiento de datos ovd líneas de asistencia, observatorio de género.
acceso y análisis de datos extensivo a provincias, no solo benos aitres


en \textit{La guerra contra las mujeres}, \citeyearpar{segato2016guerra}, Rita Segato habla de la violencia sexual como algo siempre dirigido hacia cuerpos femeninos y \textit{feminizados} (resaltado propio). Con esto último quiere decir cuerpos percibidos o construidos por los abusadores como femeninos con respecto a posiciones de poder: menores, débiles, racializados, pertenecientes a disidencias sexuales. Esto se condice con datos sobre la mayor incidencia de la violencia sexual contra identidades masculinas durante la niñez y la adolescencia, es decir, en períodos en que los cuerpos y los sujetos son más vulnerables, y por lo tanto, también percibidos como feminizados \citep*{contreras2016violencia,ufem_relevamiento,ferris2002world}.

Si las denuncias de violencia sexual contra disidencias de género representan una minoría en los datos, ¿quiere decir esto que esas personas sufren menos violencia sexual?, ¿O quiere decir que, como minoría social, están subrepresentados en general y que tienen menos acceso a la justicia?

\newpage


\bibliography{bibtex_reporte.bib}

\newpage
\section*{Anexo}\label{anex}

\end{document}